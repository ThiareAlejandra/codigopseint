\documentclass{article}
\usepackage{graphicx} % Required for inserting images
\usepackage{float}
\usepackage{tabularx}
\usepackage{hyperref}

\title{EVALUACIÓN 01 - MMIE801-1}
\author{Thiare Osorio}
\date{6 de Septiembre del 2024}

\begin{document}
\maketitle
\section*{PARTE 01: Programación}
En el siguiente apartado se mostrará evidencia a través de imágenes de la funcionalidad del código realizado en Pseint. Finalmente se entregará un enlace con la escritura del código con la extensión de Pseint.
\begin{itemize}
    \item Menú principal
    \begin{figure}[H]
    \centering
    \includegraphics[width=\linewidth]{Menú completo.png}
\end{figure}  
En el menú principal se pueden ver 4 opciones diferentes que el estudiante puede elegir para que el programa ejecute.
\newpage
    \item Opción 1
    \begin{figure}[H]
    \centering
    \includegraphics[width=\linewidth]{Opcion1.png}
\end{figure}
La primera opción le permite al estudiante poder resolver una progresión aritmética.
\newpage
    \item Opción 2
    \begin{figure}[H]
    \centering
    \includegraphics[width=\linewidth]{Opcion2.png}
\end{figure}
La segunda opción le permite al estudiante poder resolver una integral definida.
\newpage
    \item Opción 3
    \begin{figure}[H]
    \centering
    \includegraphics[width=\linewidth]{Opcion3.png}
\end{figure}
La tercera opción le permite al estudiante poder resolver una ecuación cuadrática con su fórmula general.
\newpage
    \item Opción 4
    \begin{figure}[H]
    \centering
    \includegraphics[width=\linewidth]{Opcion4.png}
\end{figure}
La cuarta opción le permite al estudiante poder cerrar el programa.
\newpage
    \item Opción 5
    \begin{figure}[H]
    \centering
    \includegraphics[width=\linewidth]{Opcion5-6.png}
\end{figure}
La quinta opción no es válida, por lo que le recuerda al estudiante que solo puede elegir entre las opciones 1 y 4.
\end{itemize}
\href{https://github.com/ThiareAlejandra/codigopseint/tree/2af2e7bece1958fd25acaafc973ef68b9b4f010a}{Archivo con la extensión en Pseint} 
\newpage
\section*{PARTE 02: Relación Programas de estudio} 
El objeto matemático elegido en la opción 3 de la Parte 01 fue "Polinomio de segundo grado" y a continuación se presentará la relación con el programa de estudio de pensamiento computacional y programación.
\begin{table}[h!]
    \centering
   \begin{tabularx}{\textwidth}{|c|X|}
        \hline
        Objetivo de Aprendizaje & OA 3: Desarrollar y programar algoritmos para ejecutar procedimientos matemáticos, realizar cálculos y obtener términos definidos por una regla o patrón.\\
        \hline
        Objetivo específico & Comprender la estructura de una ecuación cuadrática, implementando un algoritmo que la resuelva utilizando la fórmula general\\
        \hline
        Habilidades & OA g: Elaborar representaciones, tanto en forma manual como digital, y justificar cómo una misma información puede ser utilizada según el tipo de representación.\\
        \hline
        Actitudes & Aprovechar las herramientas disponibles para aprender y resolver problemas\\
        \hline
        Unidad & N°2: La resolución de problemas y las máquinas\\
        \hline
        Tiempo estimado & 4 horas pedagógicas\\
        \hline
        Sugerencia de implementación & Es recomendable iniciar la clase con una revisión de las ecuaciones cuadráticas, recordando su fórmula general y el discriminante. Después, se puede guiar a los estudiantes en la creación de un algoritmo en PseInt acerca del contenido de ecuaciones cuadráticas, explicando cada paso para que puedan implementarlo y observar los diferentes tipos de soluciones que van obteniendo según los coeficientes. Tras ejecutar el código, se discuten los resultados obtenidos, relacionando el discriminante con las soluciones. Para concluir, se puede reflexionar sobre el contenido de la clase y explorar posibles mejoras en el programa, fomentando tanto el pensamiento computacional como la comprensión matemática de los estudiantes.\\
        \hline
\end{tabularx}
\end{table}
\end{document}
